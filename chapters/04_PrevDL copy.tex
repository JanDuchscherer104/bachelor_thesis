\chapter{Deep Learning Implementations for Model Order Estimation}

\section{Introduction}
Deep Learning has shown promising results in various fields of signal processing, including model order estimation.
This chapter aims to review and elaborate on the network architectures used for model order estimation, as indicated
by recent research papers.

\section{Literature Review}
Several papers have been published on the use of Deep Learning for model order estimation. For instance, a paper by
Andreas Barthelme and Wolfgang Utschick from Technical University Munich focuses on the use of neural networks for
model order selection in DoA scenarios \cite{barthelme2020}. Another paper discusses a neural network for model order
selection in signal processing \cite{Costa1995}.

\section{Network Architectures}
\subsection{Feedforward Neural Networks}
Feedforward neural networks have been used for direction-of-arrival estimation \cite{Ozanich2020}.

\subsection{Convolutional Neural Networks (CNN)}
Deep convolution networks have been applied for direction of arrival estimation with sparse prior \cite{Wu2019}.

\subsection{Recurrent Neural Networks (RNN)}
A CRNN-based method has been proposed for coherent DOA estimation with an unknown number of sources \cite{Yao2020}.

\section{Discussion}
Deep learning techniques, especially neural networks, offer a compelling alternative to traditional model order
estimation methods. However, they come with their own set of challenges, such as the need for large datasets and
computational resources.

\section{Conclusion}
Deep Learning offers a promising avenue for improving model order estimation techniques. However, more research
is needed to make these methods more robust and efficient.
\endinput
% To-Do
% [ ] Include more empirical data to validate the performance of the discussed network architectures.
% [ ] Compare the computational complexity of deep learning methods with traditional methods.
% [ ] Include code snippets or pseudo-code for implementation details.

% \begin{thebibliography}{9}
% \bibitem{Barthelme2020}
% Andreas Barthelme and Wolfgang Utschick, "A Machine Learning Approach to DoA Estimation and Model Order Selection for Antenna Arrays with Subarray Sampling," Technical University Munich, 2020.

% \bibitem{Costa1995}
% P. Costa-Hirschauer, J. Grouffaud, P. Larzabal, and H. Clergeot, "A neural network for model order selection in signal processing," in Proc. ICNN, vol. 6, Nov. 1995, pp. 3057–3061.

% \bibitem{Ozanich2020}
% E. Ozanich, P. Gerstoft, and H. Niu, "A feedforward neural network for direction-of-arrival estimation," J. Acoustical Soc. Amer., vol. 147, no. 3, pp. 2035–2048, Mar. 2020.

% \bibitem{Wu2019}
% L. Wu, Z.-M. Liu, and Z.-T. Huang, "Deep convolution network for direction of arrival estimation with sparse prior," IEEE Signal Process. Lett., vol. 26, no. 11, pp. 1688–1692, Nov. 2019.

% \bibitem{Yao2020}
% Y. Yao, H. Lei, and W. He, "A-CRNN-based method for coherent DOA estimation with unknown source number," Sensors, vol. 20, no. 8, p. 2296, Jan. 2020.
% \end{thebibliography}
