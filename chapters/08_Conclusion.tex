
\chapter{Conclusion \& Outlook}
\label{ch:conclusion_outlook}

\section{Considerations on the feasibility of eigenvalue based methods}
Considering the evidence collected in the previous chapter, it seems clear that eigenvalue based methods for \gls{moe}
lose their applicability when the model order exceeds the number of RF-paths in a sub-sampled \gls{ddf} system.
- it is unlikely that \gls{music} would perform adequately, even if the model order was estimated correctly, given that
the entire subspace decomposition loses its theoretical soundness.
- scenarios with \( N > L \) might be unlikely in practise, hence the
The overall low bias of the neural networks, can be considered as a strong indicator that improvement could if only be achieved
though the application of models with even lower complexity. This statement however conflicts with the fact that the model
architectures are the results of extensive hyperparameter optimizations; simpler models generally performed worse than
the selected architectures. \\
This contradiction solidifies the assertion that the eigenvalue based


% \end{enumerate}



\subsubsection{Replacement of the MDL}
Considering the \gls{mdl}'s underwhelming performance and strong bias for underestimation even for \( 2 \leq N \leq 3 \),
a replacement of the \gls{mdl} with a more reliable \gls{moe} method should be considered for employment in sub-sampled \gls{ddf}
systems as a mid-term goal. \\





\section{Conclusion on the EFT}

The \gls{eft} marks a paradigmatic advancement in \gls{moe}, and provides a systematic approach in maintaining a balance
between false positive and negative rates.
This is achieved through careful threshold adjustments and by directly identifying a gap between the signal and noise eigenvalues. \\
Unlike traditional parameter-free \glspl{ic}, the \gls{eft} effectively controls bias, preventing over- or underestimation.
However, this advantage comes at the cost of an increased sensitivity to noise and single eigenvalues. \\
Future work should aim to enhance the precision of the \gls{eft} by focusing on the optimization process. This process
should consider both the balance between false positive and negative rates and the overall performance of the estimator,
as well as mitigating the increased checking frequency of higher-index eigenvalues.

The \gls{eft}'s computational efficiency, notably superior to the \gls{aic} and \gls{mdl}, is highlighted by its minimal
runtime computations. However, it is noteworthy that all estimators require an eigenvalue decomposition, which overshadows
the subsequent computations in terms of complexity.

Despite its efficacy with incoherent sources, the original \gls{eft} as proposed in~\cite{eft}, is not applicable to scenarios
with coherent sources or multipath interference. % TODO: is this a limitation of all eigenvalue-based methods?

Implementation challenges, particularly the ambiguity in calculating the decay coefficient \( q \) and estimating the noise
variance, underscore the necessity for further research and optimization of the \gls{eft}.

\subsubsection{Future Improvements of the EFT}
In essence, the \gls{eft} offers a promising approach to \gls{moe} through careful threshold optimization and bias control.
Future research should not only aim to optimize threshold values and clarify implementation methods but also explore the
proposed modifications to the \gls{eft} in~\cite{costa2007} and~\cite{costa2009}. \\
The changes suggested in~\cite{costa2007} consider the iterative estimation of the noise variance \( \hat{\sigma}^2_{\eta} \).
They proposed a solution for the aforementioned ambiguity, by estimating \( \hat{\sigma}^2_{\eta} \) from the set of the
\( p \) smallest eigenvalues as per~\autoref{eq:eft_predicted_eigenvalue_corrected}, instead of the \( p + 1 \) smallest eigenvalues
as denoted in~\autoref{eq:eft_predicted_eigenvalue}.

\begin{equation}
    \hat{\lambda}_{M-p} = \gnbxeq{(P+1)}\frac{1 - q(\gnbxeq{P+1}, K)}{1 - {q(\gnbxeq{P+1}, K)}^{p + 1}} \cdot \gnbxeq{\frac{1}{p}}\sum_{i=0}^{\gnbxeq{p-1}}\lambda_{M-i}
    \label{eq:eft_predicted_eigenvalue_corrected}
\end{equation}

All changes to~\autoref{eq:eft_predicted_eigenvalue} as formalized in~\autoref{eq:eft_predicted_eigenvalue_corrected} are
encapsulated in green boxes.

Moreover,~\cite{costa2007} computed the decay coefficients \( q \) based on the value of the respective candidate noise
subspace dimension \( p \), using \( q(p+1, K) \) instead of \( q(M, K) \).
This adjustment can be implemented by replacing \( M \) with \( p+1 \) in~\autoref{eq:eft_decay_coefficient}. \\
However, this modification does not align with our observations that the decay coefficients \( q(p+1, K) \)
lead to rather unexpected exponential eigenvalue profiles, as illustrated in~\autoref{fig:eft_coeff_comparison}.

Another noteworthy modification proposed in~\cite{costa2009} is the calculation of the \gls{eft}'s thresholds.
The authors proposed that the thresholds \( \bfm{\tau} \) should be calculated based on the conditional probability of false
alarm as \( \Pr_{\text{fa}}(p) = \Pr(\widetilde{\mathcal{H}}_{p+1}| \mathcal{H}_2, \ldots, \mathcal{H}_p) \), which expresses
the probability of detecting a signal for the \( (M - p - 1) \)-th
eigenvalue conditioned on the fact that the smallest \( p \) eigenvalues are correctly identified as noise.

In addition to these modifications,~\cite{costa2007} proposed to employ forward-backward smoothing as a pre-processing step to cope with limited
numbers of snapshots \( K \) by virtually doubling the number of available snapshots without sacrificing array aperture.
\begin{equation}
    \bfm{Z}= \left[\begin{array}{ll} \mathbf{X} & \mathbf{\Pi}_M \mathbf{X}^+ \mathbf{\Pi}_K\end{array}\right], \quad \bfm{X} \in \mathbb{C}^{M \times K}, \:\bfm{Z} \in \mathbb{C}^{M \times 2 K}
\end{equation}

In this equation, \( \Pi_n \) denotes the \(n \times n \) exchange matrix, which contains ones on its anti-diagonal and
zeros elsewhere.~\( \bfm{X} \) refers to the measurement matrix as defined in~\autoref{eq:MeasMatSingle}, and \( \bullet^+ \)
signifies the Moore-Penrose pseudo inverse of \( \bullet \).

%%%%%%%%%%%%%%%%%%%%%%%%%%%%%%%%%%%%%%%%%%%%%%%%%%%%%%%%%%%%%%%%%%%%%%%%%%%%%%%%%%%%%%%%%%%%%%%%%%%%%%%%%%%%%%%%%%%%%%%%
\section{Conclusion on the Explored DNNs}
Various forms of input data and preprocessing strategies have been considered in the context of this thesis.
Without considerations with respect to the theoretical soundness of the subspace decomposition, the eigenvalues \( \bfL \)
emerged as the most suitable input, offering an optimal balance between information encapsulation and computational
feasibility.

The thesis provided insights into numerous architectural nuances of the three explored \glspl{dnn} and their impact on
model performance and computational cost.

The \glspl{cnn} have demonstrated balanced and robust performance, highlighting the effects of normalization layers
on model generalization and the advantages of \gls{prelu} over \gls{relu} activations. \\
Their performance and training dynamics might be improved by employing residual connections, bypassing the
\( \{\texttt{Conv1D} \rightarrow \texttt{BatchNorm1D} \rightarrow \texttt{PReLU}\} \) layers, which might provide
some gain without increasing the model's complexity prohibitively. This is of course only possible, for layers whose
number of output channels matches the number of input channels. \\

The exploration of \glspl{mlp} has underscored the importance of model depth in capturing essential features and the potential
efficacy of simple architectural features such as the \gls{mlp}'s skip connections.
The feed forward model has shown the worst performance among the explored models, as well as the strongest susceptibility
to overfitting, but has had slight disadvantages during training and evaluation, as was pointed out in~\autoref{sec:mlp}. \\

The recurrent neural networks have shown the most promising results and have the highest potential for improvement.\\
The best overall accuracy has been achieved by a \gls{gru} based \gls{rnn} with a mostly unexplored hyperparameter space.
The \glspl{rnn}' suitability for \gls{fpga} implementations due to their low parameter count, which emerges from their
parameter sharing over time, has been highlighted. \\
The implementations of the bidirectional \glspl{rnn} forward the outputs of each time step to the subsequent fully connected
layer for the final classification, which presumably results in a reduced information density in the feed forward layer's input.
This, supposedly, should allow easy pruning of the network by removing the least significant connections in the final layer or
by forwarding only a selection of the output neurons to the final layer. \\


The low variance in the performances of the different neural networks implies that any further improvement would come
at a substantial cost. This implies that subsequent optimization efforts should be focused on reducing the complexities
of the models while maintaining their performance, rather than on improving their performance.

The performance evaluation further revealed relatively high variances in the models' performance, indicating that the challenges
in eigenvalue-based \gls{moe} extend beyond overfitting, implicating the intrinsic limitations of input data in encoding
the coveted model order information.


\paragraph{Outlook}
Looking ahead, the study suggests several critical areas for future research to further harness the capabilities of DNNs
in model order estimation:
\begin{itemize}
    \item Optimizing regularization strategies, including dropout and L2 regularization, to refine their application for
    improved model generalization and overfitting mitigation.
\end{itemize}


%%%%%%%%%%%%%%%%%%%%%%%%%%%%%%%%%%%%%%%%%%%%%%%%%%%%%%%%%%%%%%%%%%%%%%%%%%%%%%%%%%%%%%%%%%%%%%%%%%%%%%%%%%%%%%%%%%%%%%%%
\section{Outlook on Improvements to the Data Model}
\subsubsection{Multi-Path Propagation}
The inclusion of coherent signals into the data model would provide a valuable enhancement to generate a greater variety
of possible signal scenarios. This would further close the gap between reality and simulation environment and would allow
future deep learning models to learn more features with relevance in practical applications.

\subsubsection{Noise Floor}
The utilized Python environment does currently not support varying noise floors and introduces randomness in terms of magnitues
only through the \glspl{snr}. Considering the strong
relation between the noise floor and the behavior of the eigenvalues, this feature should be considered as a future extension.

\subsubsection{Varying Numbers of Snapshots}
Including \( K \) as a new possible parameter into the Mote-Carlo simulation, which is being used to generate the datasets,
instead of generating datasets with varying numbers of snapshots, such as \( \DKvar \), instead of wrapping the modules
for dataset generation and performing a sweep over \( K \) followed by a subsequent concatenation of the data-frames, should
be considered to improve reproducibility.

\section{Outlook on Covariance Matrix-based DNNs}

Future research should focus on developing covariance matrix-based networks specifically tailored for both \gls{moe} and
\gls{doa} estimation in \gls{ddf} systems with sub-array sampling. \\
These networks should utilize the \( J \) sub-array covariance matrices \( \C^{j} \in \mathbb{C}^{L \times L} \) as input data,
aiming to leverage intrinsic dependencies between sub-arrays and mitigating information loss due to the sub-sampling process
as formulated in~\autoref{eq:sub_sampled_covmat}. \\
\begin{equation}
    \bfm{\Gamma} \in \mathbb{R}^{\B \times J  \times 2 \times \frac{L(L+1)}{2}}
    \label{eq:cov_tensor1}
\end{equation}

The tensor \( \bfm{\Gamma} \) in~\autoref{eq:cov_tensor1} represents the structure of the input data. \\
The first dimension corresponds to the batch size \( B \).
The second dimension corresponds to each of the \( J \) sub-array covariance matrices.
The tensor has two channels for the real and imaginary parts of the covariance matrices --
\( \left[ \Re\{\bullet\} \; \Im\{\bullet\} \right] \).
Lastly, the elements from the upper triangle and diagonal of the \(L \times L\) covariance matrices are extracted to reduce redundancy -- \( \bfm{C}_{ij} = \bfm{C}_{ji} \).
This preprocessing step aims to reduce the networks' complexity and proneness to overfitting but comes at the cost of yielding
a tensor which does not exhibit a rectangular shape anymore. Utilizing convolutional layers in the initial layers of the network architecture
would thus require padding. Nonetheless, the resulting heterogeneous information distribution might also be rather non-ideal. \\
Reshaping the reduced spatial dimension into a rectangular shape would be another option; however,
this would corrupt the spatial correlation within the covariance matrices. \\
Hence, various strategies for handling the input data should be explored. \\
Additionally, these networks could be extended to utilize the covariance matrices from all \( K \) snapshots, thus providing
a more comprehensive representation of the signal environment by capturing temporal dependencies and mitigating the negative
effects arising from averaging over the snapshots. \\

\begin{equation}
    \bfm{\Gamma}' \in \mathbb{R}^{\B \times K \times J \cdot 2 \cdot \frac{L(L+1)}{2}}
    \label{eq:cov_tensor2}
\end{equation}

The tensor \( \bfm{\Gamma}' \) in~\autoref{eq:cov_tensor2} could be input into a recurrent neural network, treating
the second dimension as the temporal dimension and \( J \cdot 2 \cdot \frac{L(L+1)}{2} \) as the input size of the
recurrent layer. \\
A data representation that adheres to the different dimensions as per~\autoref{eq:cov_tensor1} might be more appropriate
for the task at hand and should be able to better leverage weight sharing, but it would not be compatible with standard recurrent layers.\\
To overcome this, it would be beneficial to investigate the effectiveness of initial convolutional layers, perhaps using
asymmetric convolutional kernels, such as \( 1 \times 3 \) and \( 3 \times 1 \), spanning over the \( J \cdot 2 \) input
channels of the reduced and padded or ``full'' \( L \times L \) covariance matrices. \\
Alternatively, the dimension accounting for the different sub-arrays could be treated independently, and the
results could then be concatenated -- the kernels would only extend over \( 2 \) input channels.

Considering the increased complexity of the data, the complexity, and sophistication of potential \glspl{dnn}, would have to
be scaled accordingly to allow a sufficient feature extraction and subsequent prediction. \\
Given that the entirety of the coveted information for both \gls{doa} estimation and \gls{moe} can be extracted from the individual sub-array covariance matrices
\( \C^{j<k>} \), it would be worthwhile to consider networks whose task is not only model order estimation, but also \gls{doa}
estimation. \\
This could be achieved by utilizing a mutual backbone for feature extraction, followed by two distinct networks for
the distinct tasks. The model order estimate might then be embedded in the \gls{doa} estimation network.

Other potential forms of input data could be explored:
\begin{itemize}
    \item The number of snapshots \( K \) could allow application of the same network and weights to scenarios with different numbers of snapshots.
    \item The estimated noise variance \( \hat{\sigma}^2_{\eta} \) could provide useful information to the network.
\end{itemize}


\section{Outlook: Optimized Training Data Utilization}
\label{subsub:optimized_training_data_utilization}
Training the final models on the \(\num{1.2e6}\) samples from \( \DMain_{(\text{train})} \) with a batch size
\( \B = 512 \) revealed suboptimal training dynamics.
Through the high cardinality of the training set, most models managed to achieve most of their training progress in the first epoch,
whereas the subsequent epochs yielded only marginal performance improvements.\\

The use of synthetic data offers a unique advantage in training \glspl{dnn}, allowing for the continuous
introduction of previously unseen data throughout the training process.
This approach eliminates the need for repetitive epochs traditionally required to achieve convergence, enabling a more
dynamic and efficient learning process. \\
However, distributing the training data across multiple ``epochs'' could allow for the effective use of learning rate
schedulers and other validation-based callbacks, such as early stopping. Such a strategy would enable a more nuanced
exploration of the parameter space, potentially leading to more robust and generalizable model performance.
Moreover, the employment of normalization layers independent of batch size, coupled with a reduction in batch size, might
synergize to enhance the model's learning process.
Smaller batch sizes introduce variability in the gradient estimations, acting as a form of regularization that can
inadvertently aid in better generalization to unseen data by preventing the model from learning noise present in the
training set too meticulously. \\
Additionally, smaller batches facilitate more frequent updates to the model's parameters, granting it numerous opportunities
to adjust and refine its learning based on a broader spectrum of data instances within the training set.


\section{Outlook: Utilization of Real-World Data}
The generalizability of the introduced deep learning methodologies to real-world scenarios warrants thorough investigation.
Currently, the absence of labeled real-world datasets poses a significant challenge to validating the practical applicability
of the methods presented in this thesis. \\
Furthermore, exploring the real-world relevance of the thesis's findings, including the
impact of \gls{snr} on \gls{moe} performance, as well as yet unexplored effects of limited resolutions and
the use of fixed-point arithmetic, are essential for the development of reliable and accurate \gls{moe} methodologies.

Exploring auxiliary \gls{moe} techniques that remain theoretically sound in sub-sampled contexts where the model order
surpasses the number of available RF-paths could pave the way for their application in analyzing recorded real-world data for
\gls{doa} estimation and model order determination. \\
These predictions of these presumably more complex methods could be performed in post-processing, thus allowing the acquisition
of ``labeled'' real-world data in a semi-supervised manner. \\
Possible methods are the conventional \gls{mle}, as well as more sophisticated methods such as the \gls{sml}, \gls{gls}, or
\gls{ssr} methods. \\
The subsequent development of auxiliary deep learning models such as transformer-based networks, which could be trained on synthetic data
and real-world data, labeled by the aforementioned methods, could provide a powerful tool for generating real-world datasets
with higher-quality labels. \\
The eventual acquisition of labeled real-world data, where the signal model is truly known a priori, could then be utilized
for testing these auxiliary methods. \\
These computational expensive methods could than be used to generate significantly larger datasets for training and testing
of more complex \glspl{dnn}.